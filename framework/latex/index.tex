\subsubsection*{Table of Contents}

\hyperlink{index_IN}{Introduction} \par
 \hyperlink{index_FE}{Features} \par
 \hyperlink{index_MK}{Building G\-E\-N\-E\-R\-I\-C\-\_\-\-S\-E\-R\-V\-E\-R} \par
 \hyperlink{index_WI}{W\-I\-N\-D\-O\-W\-S server startup} \par
 \hyperlink{index_LI}{L\-I\-N\-U\-X server startup} \par
 \hyperlink{index_SE}{Server configuration} \par
 \hyperlink{index_AR}{Architecture} \par
 \hyperlink{index_mp}{Message Structure} \par
 \hyperlink{index_au}{Author} \par
 \hyperlink{index_li}{License} \par
\hypertarget{index_IN}{}\section{Introduction}\label{index_IN}
G\-E\-N\-E\-R\-I\-C\-\_\-\-S\-E\-R\-V\-E\-R is a cross-\/platform, pluggable, extensible, secure framework for deploying C++ plug-\/ins. Multi-\/threaded framework can support multiple plug-\/ins. G\-E\-N\-E\-R\-I\-C\-\_\-\-S\-E\-R\-V\-E\-R makes extensive use of pluggable shared libraries to 'plugin' specific functionality to the framework. Framework dynamically loads,unloads plug-\/ins and dispatches each plug-\/in in a separate thread.

Plug-\/ins would just focus on implementing business logic, framework handles plug-\/in lifecycle management, network management, threads management and provides a host of other functionality that any plug-\/in could utilitze.\hypertarget{index_FE}{}\section{Features}\label{index_FE}
\begin{DoxyItemize}
\item Built-\/in support for T\-L\-S encryption. A flag in generic\-\_\-server.\-conf file will enable or disable T\-L\-S/\-S\-S\-L encryption for a specific plug-\/in. X509 Certificates installed under C\-E\-R\-T sub-\/directory are used for T\-L\-S/\-S\-S\-L implementation\end{DoxyItemize}
\begin{DoxyItemize}
\item All plug-\/ins are derived classes from a base class G\-E\-N\-E\-R\-I\-C\-\_\-\-P\-L\-U\-G\-I\-N. Lot of functionality required for plug-\/ins are implemented in G\-E\-N\-E\-R\-I\-C\-\_\-\-P\-L\-U\-G\-I\-N\end{DoxyItemize}
\begin{DoxyItemize}
\item Plug-\/in specific conf files. Configuration parameters specific for a plug-\/in should be configured in a conf file residing in plug-\/in specific directory\end{DoxyItemize}
\begin{DoxyItemize}
\item Multiple instances of framework can be running simultaneously, serving different group of plug-\/ins. Each instance would have a different process name\end{DoxyItemize}
\begin{DoxyItemize}
\item Framework managed log file, with log file rollover capability. Log can be redirected to 'syslog' on Linux and to Eventlog on Windows\end{DoxyItemize}
\begin{DoxyItemize}
\item Framework can authenticate all clients via X509 certificates. V\-E\-R\-I\-F\-Y\-\_\-\-C\-L\-I\-E\-N\-T option in main framework conf file can be used to disable/enable T\-L\-S certificate validation of client. Default is to disable T\-L\-S certificate validation\end{DoxyItemize}
\begin{DoxyItemize}
\item Framework can authorize all clients by enforcing clients to send $<$\-P\-L\-U\-G\-I\-N\-\_\-\-N\-A\-M\-E$>$ as part of request payload. This is done by setting V\-A\-L\-I\-D\-A\-T\-E\-\_\-\-P\-L\-U\-G\-I\-N flag to true for a specific plug-\/in in generic server configuration file.\end{DoxyItemize}
\begin{DoxyItemize}
\item Framework can support / service multiple plug-\/ins of same plug-\/in\-\_\-type on one port. This is termed as an 'Alias' plug-\/in. An alias plug-\/in should be of same plug-\/in\-\_\-type, have plugin shared library path and T\-L\-S flag as the primary plug-\/in. The first plug-\/in in generic\-\_\-server.\-conf file for any plug-\/in\-\_\-type is considered as the primary plug-\/in.\end{DoxyItemize}
\par
 \par
 \hypertarget{index_MK}{}\section{Building G\-E\-N\-E\-R\-I\-C\-\_\-\-S\-E\-R\-V\-E\-R}\label{index_MK}
\hypertarget{index_BL}{}\subsection{Linux}\label{index_BL}
G\-E\-N\-E\-R\-I\-C\-\_\-\-S\-E\-R\-V\-E\-R has been packaged as an Autoconf package. You could just\-: 
\begin{DoxyPre}
        \$ ./configure
        \$ make
        \$ make install
\end{DoxyPre}
 to build and install the framework. The following components would be built\-: \begin{DoxyItemize}
\item libgeneric\-\_\-plugin.\-1.\-0.\-so \par
 This is the shared library for generic\-\_\-plugin. \par
 \par
 \item \hyperlink{classgeneric__server}{generic\-\_\-server} \par
 This is the \hyperlink{classgeneric__server}{generic\-\_\-server} executable. \par
 \par
 \item libsample.\-1.\-0.\-so \par
 This is the shared library for sample plug-\/in. \par
 \par
 \item sample\-\_\-client \par
 This is a sample client program that can be used to interface with G\-E\-N\-E\-R\-I\-C\-\_\-\-S\-E\-R\-V\-E\-R. \par
 \par
 \end{DoxyItemize}
\hypertarget{index_BW}{}\subsection{Windows}\label{index_BW}
A Visual Studio solution is available to build G\-E\-N\-E\-R\-I\-C\-\_\-\-S\-E\-R\-V\-E\-R on Windows. This is generic\-\_\-server.\-sln under W\-I\-N\-D\-O\-W\-S/generic\-\_\-server directory. Visual Studio 2010 has been used to develop and test on Windows. This solution has five projects\-: \begin{DoxyItemize}
\item generic\-\_\-plugin \par
 Builds generic\-\_\-plugin.\-dll \par
 \par
 \item \hyperlink{classgeneric__server}{generic\-\_\-server} \par
 Builds generic\-\_\-server.\-exe \par
 \par
 \item sample\-\_\-plugin \par
 Builds sample\-\_\-plugin.\-dll When building on Windows, please start off by building generic\-\_\-plugin project. generic\-\_\-plugin.\-lib is required by \hyperlink{classgeneric__server}{generic\-\_\-server} and sample\-\_\-plugin projects. \par
 \par
 \item sample\-\_\-client \par
 This is a sample client program that can be used to interface with G\-E\-N\-E\-R\-I\-C\-\_\-\-S\-E\-R\-V\-E\-R. \par
 \par
 \item Generic\-Server \par
 This is the Windows installer project. It creates an .msi file. \par
 \par
 \end{DoxyItemize}
\hypertarget{index_WI}{}\section{W\-I\-N\-D\-O\-W\-S server startup}\label{index_WI}
\begin{DoxyVerb}  GENERIC_SERVER is a true Windows service. Generic_Server cleanly handles Windows service events. Service
  Manager can be used to have fine-grained control over generic_server process.
\end{DoxyVerb}
\hypertarget{index_IS}{}\subsection{Install new service}\label{index_IS}
\begin{DoxyVerb}     Edit generic_server.conf file in install directory and ensure correct path is configured for different files and plug-ins.
 Assuming framework is installed under C:\generic_server directory, the following command should be used to install GENERIC_SERVER
 service:
     C:\generic_server>generic_server -i <SERVICE_NAME> <INSTANCE_SPECIFIC_CONF_FILE>

     Example:
     C:\generic_server>generic_server -i MAIN_SERVER  generic_server.conf

  Once the above step is done, a service named MAIN_SERVER will be created on Windows.
\end{DoxyVerb}
\hypertarget{index_VE}{}\subsection{Verify service Installation}\label{index_VE}
\begin{DoxyVerb}  Go to Windows Service Manager and confirm a service named <SERVICE_NAME> is installed.

  Start --> Control Panel --> Administrative Tools --> Services
\end{DoxyVerb}
\hypertarget{index_TE}{}\subsection{Test the Server}\label{index_TE}
\begin{DoxyItemize}
\item Go to service manager and start one or more G\-E\-N\-E\-R\-I\-C\-\_\-\-S\-E\-R\-V\-E\-R service(s) \item Use sample\-\_\-client program to connect to configured port and send request to framework / plug-\/in being tested. 
\begin{DoxyPre}
        Example:
        \$ ./sample\_client x.x.x.x 60103 TEST\_PLUGIN "Hello from sample\_client.." 2 1
\end{DoxyPre}
 \end{DoxyItemize}
\hypertarget{index_LI}{}\section{L\-I\-N\-U\-X server startup}\label{index_LI}
\begin{DoxyVerb}    Run the command:

    # generic_server <INSTANCE_NAME> <INSTANCE_SPECIFIC_CONF_FILE>

    Example:
    # generic_server MAIN_SERVER generic_server.conf
\end{DoxyVerb}


\par
 \par
 \hypertarget{index_SE}{}\section{Server configuration}\label{index_SE}

\begin{DoxyPre}
        A sample server configuration file is shown below:\end{DoxyPre}



\begin{DoxyPre}        LOG\_FILE=/tmp/TEST/generic\_server.log
        #LOG\_FILE=SYSLOG
        \# This is the seconds interval a thread will do idle wait client
        TIMEOUT\_SECONDS=300
        \# Max. concurrent threads the will framework will spawn for all devices
        MAX\_THREADS=100
        \# next 2 entries are self-explanatory
        MAX\_LOG\_SIZE\_IN\_MB=100
        MAX\_LOGS\_SAVED=5
        \# TLS certificate in PEM format, if TLS is enabled for any device
        SERVER\_CERTIFICATE=/tmp/TEST/CERT/generic\_server.crt
        \# TLS private\_key in PEM format, if TLS is enabled for any device
        RSA\_PRIVATE\_KEY=/tmp/TEST/CERT/generic\_server.key
        \# CA cert file
        CA\_CERTIFICATE=/tmp/TEST//CERT/CAfile.pem
        #VERIFY\_CLIENT option will validate client TLS certificate. Default is
        #not to validate certificates. 
        VERIFY\_CLIENT=0
        \# COMMAND\_PORT should be unique per instance. Please set any free port 
        COMMAND\_PORT=10000
        \# PLUGINS\_PATH set the base DIR of plugins shared library / DLL
        PLUGINS\_PATH=/tmp/TEST/plugins
        \#
        #PLUGIN\_NAME|PLUGIN\_TYPE|TCP\_PORT|PLUGIN\_NUMBER|PLUGIN\_SHARED\_LIBRARY|TLS\_FLAG|PLUGIN\_SPECIFIC\_CONF\_FILE
        #PLUGIN\_SPECIFIC\_CONF\_FILE and PLUGIN LIB path is relative to server install directory
        \#
        SAMPLE1|SAMPLE\_TYPE|60103|1|sample/libsample.1.0.so|1|sample/sample.conf
        SAMPLE2|SAMPLE\_TYPE|60103|2|sample/libsample.1.0.so|1|sample/sample.conf
        SAMPLE3|SAMPLE3\_TYPE|60105|3|sample3/libsample.1.0.so|1|sample3/sample3.conf
        SAMPLE4|SAMPLE4\_TYPE|60106|4|sample4/libsample.1.0.so|1|sample4/sample4.conf
\end{DoxyPre}
 \par
 \par
 \hypertarget{index_AR}{}\section{Architecture}\label{index_AR}
\begin{DoxyVerb}   GENERIC_SERVER has the following main components:
\end{DoxyVerb}
\hypertarget{index_FR}{}\subsection{Framework}\label{index_FR}
\begin{DoxyVerb}   This is a singleton class and provides framework functionality. Source files are in: framework directory.
\end{DoxyVerb}
\hypertarget{index_GE}{}\subsection{Generic Plugin}\label{index_GE}
\begin{DoxyVerb}   This component provides functionality that are common across plug-ins. Source files are in
   generic_plugin directory. Framework would instantiate and load objects of type GENERIC_PLUGIN 
   to framework from plugin shared library. Generic_plugin is packaged into a shared library:
   libgeneric_plugin.so or generic_plugin.dll.
\end{DoxyVerb}
\hypertarget{index_PL}{}\subsection{Plug-\/ins}\label{index_PL}
\begin{DoxyVerb}   Plug-ins are where business logic gets implemented. All plug-ins should derive from class GENERIC_PLUGIN
   and should have a function named create_instance(). A sample plug-in is in directory: plugins/sample.
\end{DoxyVerb}
\hypertarget{index_Control}{}\subsection{Control}\label{index_Control}
Master program that bootstraps G\-E\-N\-E\-R\-I\-C\-\_\-\-S\-E\-R\-V\-E\-R. Source is in control.\-cpp. Listens and does 'select()'on all sockets configured in generic\-\_\-server.\-conf file. Spawns a thread for each plug-\/in and runs message loop for that plug-\/in. \par
 \par
 \hypertarget{index_DY}{}\section{Dynamically re-\/load plug-\/ins}\label{index_DY}
\begin{DoxyVerb}Generic_Server has been designed to easily enable addition or deletion of
plug-ins to the framework without restartng the framework:
\end{DoxyVerb}
\hypertarget{index_wi}{}\subsection{W\-I\-N\-D\-O\-W\-S}\label{index_wi}
\begin{DoxyVerb}    Add or remove plug-ins from generic_server.conf file, then go to service manager and 
    'pause' and 'resume' the service <SERVICE_NAME>. Check out log file to ensure server has
    picked up the changes.
\end{DoxyVerb}
\hypertarget{index_lii}{}\subsection{Linux}\label{index_lii}
\begin{DoxyVerb}    Add or remove plug-ins from generic_server conf file and then send signal SIGHUP to 
    generic_server process via 'kill' command. Check out log file to ensure server has
    picked up the changes.
\end{DoxyVerb}
\hypertarget{index_mp}{}\section{Message Structure}\label{index_mp}
G\-E\-N\-E\-R\-I\-C\-\_\-\-S\-E\-R\-V\-E\-R has been designed to provide flexibility to plug-\/ins on messaging structure and protocol. G\-E\-N\-E\-R\-I\-C\-\_\-\-S\-E\-R\-V\-E\-R today implements a very simplified message structure shown below\-: \hypertarget{index_hdr}{}\subsection{Message Header}\label{index_hdr}
\begin{DoxyItemize}
\item S\-I\-G\-N\-A\-T\-U\-R\-E -\/ 4 bytes S\-I\-G\-N\-A\-T\-U\-R\-E for all messages \item P\-A\-Y\-L\-O\-A\-D\-\_\-\-S\-I\-Z\-E -\/ 4 bytes network byte order \item D\-A\-T\-A -\/ P\-A\-Y\-L\-O\-A\-D\-\_\-\-S\-I\-Z\-E data \end{DoxyItemize}
\hypertarget{index_subhdr}{}\subsection{Data payload}\label{index_subhdr}
\begin{DoxyItemize}
\item P\-L\-U\-G\-I\-N\-\_\-\-N\-A\-M\-E\-\_\-\-L\-E\-N -\/ 4 bytes network byte order (optional) \item P\-L\-U\-G\-I\-N\-\_\-\-N\-A\-M\-E -\/ Size as defined by $<$\-P\-L\-U\-G\-I\-N\-\_\-\-N\-A\-M\-E\-\_\-\-L\-E\-N$>$ (optional) \item A\-P\-P\-L\-I\-C\-A\-T\-I\-O\-N\-\_\-\-D\-A\-T\-A -\/ P\-A\-Y\-L\-O\-A\-D\-\_\-\-S\-I\-Z\-E-\/(\mbox{[}4+\-P\-L\-U\-G\-I\-N\-\_\-\-N\-A\-M\-E\-\_\-\-L\-E\-N\mbox{]})\end{DoxyItemize}
If V\-A\-L\-I\-D\-A\-T\-E\-\_\-\-P\-L\-U\-G\-I\-N is set to 1 in framework configuration file, then all client requests should have P\-L\-U\-G\-I\-N\-\_\-\-N\-A\-M\-E\-\_\-\-L\-E\-N and P\-L\-U\-G\-I\-N\-\_\-\-N\-A\-M\-E within the message.\hypertarget{index_ssl}{}\section{Open\-S\-S\-L}\label{index_ssl}
\hypertarget{index_sslwi}{}\subsection{W\-I\-N\-D\-O\-W\-S}\label{index_sslwi}
\begin{DoxyVerb}   OpenSSL 1.0 was used to develop and test GENERIC_SERVER for Windows. Please download and install
   OpenSSL windows development package from: 
   http://www.shininglightpro.com/download/Win32OpenSSL_Light-1_0_0a.exe
\end{DoxyVerb}
 Once Open\-S\-S\-L is installed, please ensure path to Open\-S\-S\-L include and lib directories is correctly configured in all Visual Studio projects \hypertarget{index_sslli}{}\subsection{Linux}\label{index_sslli}
\begin{DoxyVerb}   Please use platform specific tools to install OpenSSL development package for Linux.
\end{DoxyVerb}
\hypertarget{index_au}{}\section{Author}\label{index_au}
This has been developed by  Suraj Vijayan \href{mailto:suraj@broadcom.com}{\tt suraj@broadcom.\-com}.\hypertarget{index_li}{}\section{License}\label{index_li}
\begin{DoxyVerb}   This program is free software; you can redistribute it and/or
   modify it under the terms of the GNU Lesser General Public
   License as published by the Free Software Foundation; either
   version 2.1 of the License, or (at your option) any later version.

   This library is distributed in the hope that it will be useful,
   but WITHOUT ANY WARRANTY; without even the implied warranty of
   MERCHANTABILITY or FITNESS FOR A PARTICULAR PURPOSE.  See the GNU
   Lesser General Public License for more details.

   You should have received a copy of the GNU Lesser General Public
   License along with this library; if not, write to the Free Software
   Foundation, Inc., 51 Franklin Street, Fifth Floor, Boston, MA  02110-1301  USA\end{DoxyVerb}
 